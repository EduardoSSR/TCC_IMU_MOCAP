\begin{resumo}[Abstract]
 \begin{otherlanguage*}{english}
 	
 	The objective of this work is the development of an IMU for the acquisition of human body movements and positions to assist researsh in the area of injury prevention. There are many sensors capable of assisting in the study of body movement, but most of them are not accessible or do not to detail certain useful parameters(e.g. speed and acceleration). Injury prevention is very important for high-performance athletes, because of preventing injuries, athletes can stay longer in training and achieve better marks. The MPU6050 is an inertial sensor made whit MEMS technology, integrating on a single chip, 3 acceleration sensors and 3 angular speed sensors. This sensor communicates by I2C protocol, which allows the use of the Arduino platform, which has a programmable microcontroller. This microcontroller communicates serially to the computer via a USB cable. The Python programming language can communicate with the Arduino through a serial communication protocol and thus allow the data acquired by the MPU6050 sensor to be stored on the computer. The data can be manipulated to obtain various information (e.g. linear acceleration, angular velocity, positions and angles). So, a functional prototype was done with a simple interface for the use of the researchers of the laboratory LIS of the UnB/FGA. Needing of future tests to verify the accuracy and precision of the IMU.

   \vspace{\onelineskip}
 
   \noindent 
   \textbf{Key-words}: latex. abntex. text editoration.
 \end{otherlanguage*}
\end{resumo}
