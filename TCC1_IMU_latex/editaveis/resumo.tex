\begin{resumo}
 
 \noindent
 O objetivo deste trabalho é o desenvolvimento de um \textit{Inertial Measurement Unit} (IMU) para aquisição de movimentos e posições corporais em humanos, a fim de auxiliar em pesquisas na área de prevenção de lesão. Existem muitos sensores capazes de auxiliar no estudo do movimento do corpo, mas a maioria deles não é acessível ou não conseguem detalhar certos parâmetros úteis, por exemplo, velocidade e aceleração. A prevenção de lesão é muito importante para atletas de alto desempenho, pois prevenindo lesões os esportistas conseguem se manter mais tempo treinando e atingir melhores marcas. O MPU6050 é um sensor inercial feito a partir de tecnologia \textit{Micro Electro Mechanical System} (MEMS), que integra em um único \textit{chip}, 3 sensores de aceleração e 3 sensores de velocidade angular. Este sensor comunica-se por protocolo I2C, o que permite a utilização da plataforma Arduino, que possui um microcontrolador programável. Este microcontrolador se comunica serialmente ao computador, por meio de um cabo \textit{Universal  Serial Bus} (USB). A linguagem de programação \textit{Python} pode ser comunicar com o Arduino por meio de um protocolo de comunicação serial e assim possibilitou que os dados adquiridos pelo sensor MPU6050 fossem armazenados no computador. Os dados podem ser manipulados para obtenção de diversas informações: aceleração linear, velocidade angular, posições e ângulos. Então, foi feito um protótipo funcional com uma \textit{interface} simples para utilização dos pesquisadores do laboratório LIS da UnB/FGA, necessitando de futuros testes para verificar a precisão e acurácia do IMU. 
 
 \vspace{\onelineskip}
 \textbf{Palavras-chaves}:IMU, MPU6050, Prevenção de Lesão.
\end{resumo}
