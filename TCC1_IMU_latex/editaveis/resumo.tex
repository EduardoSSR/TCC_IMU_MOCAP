\begin{resumo}
 
 \noindent
 O objetivo deste trabalho é o desenvolvimento de uma Unidade de Medição Inercial (IMU) para aquisição de parâmetros de movimento e posições corporais em humanos, a fim de auxiliar pesquisas futuras na área de prevenção de lesões musculoesqueléticas. Existem muitos sensores capazes de adquirir dados sobre  movimento, mas a maioria não são acessíveis ou não conseguem detalhar certos parâmetros úteis, por exemplo, velocidade e aceleração. A prevenção de lesão é muito importante para atletas de alto desempenho, pois os esportistas conseguem permanecer mais tempo treinando e atingem melhores marcas. O MPU6050 é um sensor inercial feito a partir de tecnologia \textit{Micro Electro Mechanical System} (MEMS), que integra em um único \textit{chip}, 3 sensores de aceleração e 3 sensores de velocidade angular. Este sensor comunica-se por protocolo I2C, o que permite a utilização das plataformas Arduíno e Wemos, que possuem microcontroladores programáveis. O microcontrolador do Arduíno se comunica serialmente ao computador, por meio de um cabo \textit{Universal  Serial Bus} (USB). O do Wemos pode se comunicar via \textit{wireless} com um protocolo \textit{web}. A linguagem de programação \textit{Python} pode se comunicar com o Arduino por meio de protocolo de comunicação serial, ou com o Wemos, com protocolo HTTP, e assim possibilitar que os dados adquiridos pelo sensor MPU6050 sejam armazenados no computador. Os dados podem ser manipulados para obtenção de diversas informações: aceleração linear, velocidade angular, posições e ângulos. Então, foi feito um protótipo funcional com uma \textit{interface} via terminal, para utilização dos pesquisadores do laboratório LIS da UnB/FGA, necessitando de futuros testes com movimentação em humanos para validar a precisão e acurácia do IMU. 
 
 \vspace{\onelineskip}
 \textbf{Palavras-chaves}:IMU, MPU6050, Movimento Humano.
\end{resumo}
