%\part{Introdução}

\chapter[Introdução]{Introdução}
%\addcontentsline{toc}{chapter}{Introdução}

O interesse em analisar por meio de conceitos físicos o movimento humano é bastante antigo. Estudos clássicos como os de Aristóteles deixam claro que a relevância de analisar o movimento por meio de ensaio físico data do século III a.C. Porém, mesmo com o estudo do movimento sendo antigo, a Biomecânica se consolidou como uma ciência e disciplina acadêmica no Brasil recentemente. Historicamente falando, apenas na década de 60 com influência do governo da República Federal da Alemanha que apoiou algumas universidades brasileiras a introduzir a Biomecânica nos cursos de Educação Física \cite{Acquesta2008}.

É essencial, para associar medidas físicas ao movimento,  estudos relacionados à cinética e cinemática do movimento precisos. Com os dados adquiridos é possível realizar diagnóstico de desempenho em atletas, pesquisas para prevenção de lesão musculoesquelético e adquirir melhor entendimento do movimento humano como um todo \cite{mcginnis2013}.

Dessa forma, o estudo da biomecânica do movimento se tornou importante para desenvolver atletas em diversos esportes. No caso de um velocista, a coleta e a análise de dados sobre a velocidade, aceleração, postura e qualidade de movimento são uteis para possibilitar a evolução de seu desempenho \cite{okazaki2012}

Para realizar a coleta de dados que possibilitem aos profissionais da área de esportes, procurar por diferentes estratégia e métodos de treinamento para obtenção de melhores resultados no desempenho dos atletas, são necessários equipamentos adequados, os quais estão sendo desenvolvidos \cite{okazaki2012}.  

%Para a coleta dos dados ser realizada, são necessários equipamentos, que estão sendo desenvolvidos, que possibilitam aos profissionais da área de esportes procurar por diferentes estratégias e métodos de treinamento, para obter melhores resultados de desempenho dos atletas \cite{okazaki2012}.

As unidades de medição inercias (IMU - \textit{Inertial Measurement Units}), são um exemplo desses equipamentos, e tornaram-se ferramentas muito úteis para aquisição de informações relacionadas ao movimento corporal. Esses sensores são baratos, pequenos e permitem mobilidade quando integrados com módulos de comunicação sem fio. Porém, sua utilização necessita de um conhecimento técnico e matemático específico e complexo \cite{ober2015}.

A evolução desses sensores permitiu que a movimentação corporal humana fosse estudada em diversos ambientes e situações, sendo necessário apenas algumas unidades de sensores colocados nos pontos adequados do corpo. E, assim, os dados adquiridos são capazes de proporcionar algumas variáveis importantes como: aceleração, velocidade angular, velocidade linear, altura, direção e ângulos, todos de forma não invasiva e sem a necessidade de ambientes fechados \cite{chang2016}. A partir desses dados é possível fazer análises preditivas, diagnósticos de lesões, movimentos assíncronos de membros corporais, má postura entre outros.



\section{Objetivos}

Os principais objetivos desse trabalho de conclusão de curso são os seguintes:

\subsection{Objetivo Geral}

Desenvolvimento de um IMU para aquisição de movimentos e posições corporais em humanos a fim de auxiliar em pesquisas na área de prevenção de lesões musculoesqueléticas.

\subsection {Objetivos Específicos} 
 \begin{itemize} 
		\item Identificar quais serão parâmetros que o IMU deve ser capaz de calcular;
		%acrescentar após testes (velocidade e ângulo)
		
		\item Desenvolver e realizar testes para leitura de mais de um sensor simultaneamente;
		
		\item Desenvolver um \textit{case} para que o IMU possa ser colocado em diversas parte do corpo humano;
		
		\item Estabelecer um a comunicação sem fio entre o IMU e um computador;
		
		\item Desenvolver Kit's para utilização em pesquisas futuras do Laboratório de Informática em Saúde (LIS) da Universidade de Brasília (UnB), Faculdade Gama (FGA).
		
		
	\end{itemize}
\section{Justificativa}

O sensor IMU tem inúmeras aplicações na área da saúde e dos esportes como avaliação de postura, assimetria de movimentos, análise de marcha, movimentos específicos de esportes entre outras. É um sensor essencial para pesquisas nessas áreas. E são poucos os trabalhos que tentam tornar esse sensor mais amigável para utilização por pesquisadores de diversas áreas que não tem o conhecimento matemático e de linguagens de programação necessários para trabalhar com ele \cite{ober2015}\cite{chang2016}. 

Além disso, este sensor, exige um custo financeiro menor do que outros equipamentos utilizados nessa área de estudo, como o \textit{Motion Capture} (MOCAP), por exemplo, é um equipamento que exige um ambiente interno com espaço específico, muitas câmeras para melhor precisão e sistemas que normalmente tem um custo financeiro alto \cite{chang2016}. 

Esse sensor deverá futuramente ser utilizado para  pesquisas no laboratório LIS da UnB e auxiliar estudantes de graduação e mestrado em seus trabalhos.