\begin{apendicesenv}

\partapendices

\chapter{Código Fonte do Arduíno com um Sensor}
\label{apA}
// Programa para leitura do Módulo MPU 6050 \\
// Autor: Eduardo-ssr\\

//Carrega as bibliotecas\\
\#include<Wire.h>\\
\#include <I2Cdev.h>\\
\#include <MPU6050.h>\\

//Endereco I2C do MPU6050\\
const int MPU=0x68;  \\

//Variaveis para armazenar valores dos sensores\\
double AcX,AcY,AcZ,Tmp,GyX,GyY,GyZ;\\
char command;\\
int test = 0;\\
int c = 0;\\

void setup()\\
\{\\
Serial.begin(115200);\\
Serial.print("Digite o comando.(0,1,2 ou 3)\&$\backslash$n");\\
delay(1000);\\
Wire.begin();\\
do\{\\
if (Serial.available())\{\\

command = Serial.read();\\

if(command == '0')\{\\  
Serial.print("Comando 0 recebido.$\backslash$n");\\
c = 0;\\
test = 1;\\
delay(2);\\	
\}\\	
else if(command == '1')\{\\

Serial.print("Comando 1 recebido.$\backslash$n");\\    
c = 1;\\
test = 1;\\
delay(2); \\
\}\\
else if (command == '2')\{\\
Serial.print("Comando 2 recebido.$\backslash$n");\\    
c = 2;\\
test = 1;\\
delay(2);\\
\}\\
else if (command == '3')\{\\
Serial.print("Comando 3 recebido.$\backslash$n");\\   
c = 3;\\
test = 1;\\
delay(2);\\
\}\\
else\{\\
Serial.print("Comando invalido. Tente outro.$\backslash$n");\\ 
delay(1000);\\
test = 0;\\
\}\\
\}\\
\}while(test == 0);\\
Wire.beginTransmission(MPU);\\
Wire.write(0x6B); \\

//Inicializa o MPU-6050\\
Wire.write(c); \\
Wire.endTransmission(true);\\
\}\\
void loop()\\
\{ \\

Wire.beginTransmission(MPU);\\
Wire.write(0x3B);  // starting with register\\ 0x3B (ACCEL\_XOUT\_H)\\
Wire.endTransmission(false);\\

//Solicita os dados do sensor\\
Wire.requestFrom(MPU,14,true); \\ 

//Armazena o valor dos sensores nas variaveis correspondentes\\
AcX=Wire.read()<<8|Wire.read();  //0x3B (ACCEL\_XOUT\_H) \& 0x3C (ACCEL\_XOUT\_L)    \\ 
AcY=Wire.read()<<8|Wire.read();  //0x3D (ACCEL\_YOUT\_H) \& 0x3E (ACCEL\_YOUT\_L)\\
AcZ=Wire.read()<<8|Wire.read();  //0x3F (ACCEL\_ZOUT\_H) \& 0x40 (ACCEL\_ZOUT\_L)\\
Tmp=Wire.read()<<8|Wire.read();  //0x41 (TEMP\_OUT\_H) \& 0x42 (TEMP\_OUT\_L)\\
GyX=Wire.read()<<8|Wire.read();  //0x43 (GYRO\_XOUT\_H) \& 0x44 (GYRO\_XOUT\_L)\\
GyY=Wire.read()<<8|Wire.read();  //0x45 (GYRO\_YOUT\_H) \& 0x46 (GYRO\_YOUT\_L)\\
GyZ=Wire.read()<<8|Wire.read();  //0x47 (GYRO\_ZOUT\_H) \& 0x48 (GYRO\_ZOUT\_L)\\


//Envia valor X do acelerometro para a serial\\
Serial.print(AcX);\\
Serial.print("$\backslash$t");\\

//Envia valor Y do acelerometro para a serial\\ 
Serial.print(AcY);\\  
Serial.print("$\backslash$t");\\

//Envia valor Z do acelerometro para a serial\\
Serial.print(AcZ);\\
Serial.print("$\backslash$t");\\

//Envia valor da temperatura para a serial\\
//Calcula a temperatura em graus Celsius\\
Serial.print(Tmp);//(Tmp/340.00+36.53)\\
Serial.print("$\backslash$t");\\

//Envia valor X do giroscopio para a serial\\ 
Serial.print(GyX);\\
Serial.print("$\backslash$t");\\

//Envia valor Y do giroscopio para a serial\\   
Serial.print(GyY);\\
Serial.print("$\backslash$t");\\

//Envia valor Z do giroscopio para a serial\\ 
Serial.println(GyZ);\\
//Serial.print("$\backslash$t");\\
//Serial.println(" ");\\
// delay(100);\\

\}\\
\chapter {Código Arduino com 3 Sensores}
\label{apB}
\footnotesize
\lstinputlisting[language = Python]{editaveis/read_mpu_multiplex.c} 

\chapter {Código Wemos D1 como Ponto de Acesso}
\label{apC}
\footnotesize
\lstinputlisting[language = Python]{editaveis/acess_point_mpu.c} 

\chapter{Código do Programa em Python com um Sensor}
\label{apD}
\footnotesize
\lstinputlisting[language = Python]{editaveis/read_arduino_v5.py} 
\chapter{Código do Programa em Python com 3 Sensores}
\label{apE}
\footnotesize
\lstinputlisting[language = Python]{editaveis/Read_multiplex.py} 
\chapter{Funções de Comunicação}
\label{apF}
\footnotesize
\lstinputlisting[language = Python]{editaveis/i2cMPU6050.c} 

\chapter{Código do Programa em Python para Leitura HTML}
\label{apG}
\footnotesize
\lstinputlisting[language = Python]{editaveis/Read_HTML.py}

\end{apendicesenv}
